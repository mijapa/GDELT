 \hoffset1cm % przesunięcie poziome (przykładowo) o 1cm przeznaczone na oprawę

 \documentclass[11pt]{report}
 
 \usepackage[T1]{fontenc}
 \usepackage[utf8]{inputenc}
 \usepackage{graphicx}
 \usepackage{amsmath,amssymb,amsfonts}
 \usepackage{polski}
 \usepackage[raggedright]{titlesec}
 \usepackage{amsmath}
 \usepackage{amssymb}
 \usepackage{indentfirst}
 \usepackage{listings}
 \usepackage{hyperref}
 \usepackage[backend=biber, bibencoding=utf8, style=ieee, dashed=false, isbn=false, doi=false, sorting=anyvt]{biblatex}
 
 \addbibresource{library.bib}
 \DeclareUnicodeCharacter{229}{ę}
 \DeclareUnicodeCharacter{327}{}
 
 \pagestyle{headings}
 
 \renewcommand{\chaptername}{Rozdział}
 \renewcommand{\contentsname}{Spis treści}
 \renewcommand{\figurename}{Rys.}
 \renewcommand{\tablename}{Tab.}
 \renewcommand{\listfigurename}{Spis rysunków}
 \renewcommand{\listtablename}{Spis tabel}
 \renewcommand{\bibname}{Bibliografia}

\makeatletter
\renewcommand{\l@section}{\@dottedtocline{1}{1.5em}{2.6em}}
\renewcommand{\l@subsection}{\@dottedtocline{2}{4.0em}{3.6em}}
\renewcommand{\l@subsubsection}{\@dottedtocline{3}{7.4em}{4.5em}}
\makeatother


 \begin{document}

 \begin{titlepage}
 \centering
\includegraphics[width=0.8 \textwidth]{fig/AGH.jpg}
 \center{\scshape WYDZIAŁ INFORMATYKI, ELEKTRONIKI\\ I TELEKOMUNIKACJI\\
         Kierunek Informatyka}
 \vspace{0.03\textheight}
 \center{\scshape Michał Patyk}
 \bigskip
 \center{\LARGE\bfseries Analiza danych i wzorców dotyczących wydarzeń politycznych na podstawie informacji zgromadzonych w projekcie GDELT}
 \center{(pracownia problemowa)}
 \vspace{0.2\textheight}
 \par
 \rightline{Opiekun: dr hab. inż. Koźlak Jarosław}

 \vspace{0.1\textheight}
 \center{Kraków 2020}
 \end{titlepage}


 \tableofcontents


 \chapter{Wstęp}
 
 \section{GDELT}
 GDELT - Global Database of Events, Language, and Tone - to największa, najbardziej wszechstronna i otwarta baza danych jaka powstała. Wczesne poszukiwania prowadzące do stworzenia GDELT zostały opisane przez Philipa Schrodta w dokumencie \cite{Schrodt2010} w styczniu 2010 r. Zbiór danych jest dostępny na stronie Projektu oraz na platformie Google Cloud gdzie można z niego korzystać przez Google BigQuery \cite{BigQuery2014}. GDELT używa kodowania obserwacji konfliktów i mediacji (CAMEO) \cite{GDELTDocumentation} do rejestrowania zdarzeń. 

 
 \section{Stan aktualnej wiedzy}
 \subsection{...}
 
 \section{Motywacja}


 \chapter{Cele pracy, zakres pracy, założenia}\label{ch:cele}

 \section{Cele pracy}
 Celem niniejszej pracy jest \textbf{opracowanie koncepcji sterownika pieca kominkowego} zgodnego ze szkicem specyfikacji Web Thing API oraz \textbf{stworzenie dedykowanego oprogramowania}.

 \section{Zakres pracy}
 Zakres pracy obejmuje:
 \begin{enumerate}
 \item przegląd istniejących rozwiązań - zarówno sprzętowych jak i programowych;
 \item opracowanie koncepcji;
 \item wybór podzespołów;
 \item wykonanie prototypu na płytce stykowej;
 \item stworzenie oprogramowania;
 \item przetestowanie oprogramowania; 
 \item wykonanie prototypu na płytce uniwersalnej;
 \item wykonanie obudowy;
 \item zintegrowanie z Mozilla Gateway
 \end{enumerate}
 
 \section{Założenia i wymagania}
 
 \subsection{Wykorzystane narzędzia}
 \begin{enumerate}
 \item[•] środowisko programistyczne PyCharm 
 \item[•] język programowania Python
 \item[•] internetowe interaktywne środowisko obliczeniowe Notebook Jupyter
 \end{enumerate}
 
 \subsection{Wymagania}

 
 \section{Efekt końcowy}
 Planowanym efektem końcowym pracy będzie...
 
 \section{Dalsze kierunki rozwoju}
 W tej części zostaną opisane możliwe kierunki rozwoju pracy.
 \subsection{Analiza pod kątem COVID-19}
 \subsection{Analiza pod kątem grup etnicznych}
 \subsection{Analiza pod kątem grup religijnych}
 
 
 \chapter{Przegląd istniejących analiz}\label{ch:przeglad}
 \section{•}
  
   
  
 \chapter[Opracowanie koncepcji budowy sterownika\\ pieca kominkowego]{Opracowanie koncepcji budowy sterownika pieca kominkowego}\label{ch:koncepcja}
 
 
 \chapter{Podsumowanie}
W niniejszej części zostaną opisane wnioski z pracy według kolejności wcześniej przedstawionych rozdziałów.


 \inputencoding{utf8}
 
 \newpage
 \addcontentsline{toc}{chapter}{Książki}
 \printbibliography[title={Książki},type=book]
 
 \addcontentsline{toc}{chapter}{Artykuły}
 \printbibliography[title={Artykuły},type=article]
 
 \addcontentsline{toc}{chapter}{Prace dyplomowe}
 \printbibliography[title={Prace dyplomowe}, type=thesis]
 
 \addcontentsline{toc}{chapter}{Materiały konferencyjne}
 \printbibliography[title={Materiały konferencyjne},type=inproceedings]
 
 \addcontentsline{toc}{chapter}{Pozostałe źródła}
 \printbibliography[title={Pozostałe źródła}, nottype=article, nottype=book, nottype=inproceedings, nottype=thesis]

 \end{document}
