%! suppress = Makeatletter
%! suppress = Makeatletter
\documentclass[11pt]{report}

\usepackage[T1]{fontenc}
\usepackage[utf8]{inputenc}
\usepackage{graphicx}
\usepackage{amsmath,amssymb,amsfonts}
\usepackage{polski}
\usepackage[raggedright]{titlesec}
\usepackage{indentfirst}
\usepackage{listings}
\usepackage{hyperref}
\usepackage[backend=biber, bibencoding=utf8, style=ieee, dashed=false, isbn=false, doi=false, sorting=anyvt]{biblatex}
\usepackage{caption}
\captionsetup{%
justification=raggedright,
labelfont=bf,
singlelinecheck=off
}

\addbibresource{library.bib}

\pagestyle{headings}

\renewcommand{\chaptername}{Rozdział}
\renewcommand{\contentsname}{Spis treści}
\renewcommand{\figurename}{Rys.}
\renewcommand{\tablename}{Tab.}
\renewcommand{\listfigurename}{Spis rysunków}
\renewcommand{\listtablename}{Spis tabel}
\renewcommand{\bibname}{Bibliografia}

\makeatletter
\renewcommand{\l@section}{\@dottedtocline{1}{1.5em}{2.6em}}
\renewcommand{\l@subsection}{\@dottedtocline{2}{4.0em}{3.6em}}
\renewcommand{\l@subsubsection}{\@dottedtocline{3}{7.4em}{4.5em}}
\makeatother


\begin{document}

    \begin{titlepage}
        \centering
        \includegraphics[width=\linewidth]{fig/AGH.jpg}
        \center{\scshape WYDZIAŁ INFORMATYKI, ELEKTRONIKI\\ I TELEKOMUNIKACJI\\
        Kierunek Informatyka}
        \vspace{0.03\textheight}
        \center{\scshape Michał Patyk}
        \bigskip
        \center{\LARGE\bfseries Analiza danych i wzorców dotyczących wydarzeń politycznych na podstawie informacji zgromadzonych w projekcie GDELT}
        \center{(pracownia problemowa)}
        \vspace{0.2\textheight}
        \par
        \rightline{Opiekun: dr hab. inż. Koźlak Jarosław}

        \vspace{0.1\textheight}
        \center{Kraków 2020}
    \end{titlepage}

    \tableofcontents


    \chapter{Wstęp}
    //Wydarzenia są opisywane przez zaangażowanych aktorów (którymi mogą być państwa lub specyficzne organizacje) oraz lokalizacje i kategorie definiujące charakter wydarzeń (np. różne rodzaje negocjacji , deklaracji politycznych, zamieszki, konflikty zbrojne, itp.), co jest specyfikowane przez zestawy szczegółowych atrybutów.

    //Postawić problem
    //dlaczego chcemy realizować
    //na podstawie analiz chcemy lepiej zrozumieć specyfikę krajów
    //chęć automatycznej reakcji na to że cos sie dzieje
    //informacja o zdarzeniach nietypowych
    //jak społeczności reagują
    //patrz dyplom KKI


    \section{Cele pracy}
    Celem niniejszej pracy jest analiza wydarzeń politycznych oraz poszukiwanie występujących w nich wzorców w oparciu o dane projektu GDELT.


    \chapter{Przegląd dziedziny}
    W tym rozdziale opisany został zbiór danych GDELT, schemat kodowania CAMEO oraz przeprowadzony został przegląd istniejących analiz zbioru.


    \section{GDELT}
    GDELT - Global Database of Events, Language, and Tone - to największa, najbardziej wszechstronna i otwarta baza danych jaka powstała.
    Wczesne poszukiwania prowadzące do stworzenia GDELT zostały opisane przez Philipa Schrodta w dokumencie~\cite{Schrodt2010} w styczniu 2010 r.
    Zbiór danych jest dostępny na stronie Projektu ~\cite{gdelt} oraz na platformie Google Cloud gdzie można z niego korzystać przez Google BigQuery~\cite{BigQuery2014}.
    GDELT używa kodowania obserwacji konfliktów i mediacji (CAMEO)~\cite{GDELTDocumentation} do rejestrowania zdarzeń.
    W zbiorze znajdują się dane od 1979 roku.
    Kolejne porcje zdarzeń i ich klasyfikacja generowane sa na bieżąco każdego dnia.


    \section{CAMEO}
    CAMEO - Conflict and Mediation Event Observations - jest schematem kodowania zdarzeń.
    Został stworzony w Katedrze Nauk Politycznych Pennsylvania State University.
    Jego początki sięgają roku 2000.
    Został zaprojektowany z myślą o automatycznym kodowaniu i szczegółowym kodowaniu aktorów.

    \subsection{Zdarzenia}
    Kody zdarzeń są ujednolicone pod względem kolejności numerycznej głównych kategorii.
    Kategorie są uszeregowane rosnąco względem kooperacji od 01 do 09 oraz względem konfliktu od 10 do 20.

    \subsection{Aktorzy}
    Kody aktorów składają sie z trzech znaków.
    Elementy kodu są podzielone na szerokie kategorie, takie jak podmioty państwowe, role, regiony i grupy etniczne aktorów.


    \section{Przegląd istniejących analiz} \label{ch:przeglad}
    // artykuły analizujące, kto cytował, jakie analizy


    \chapter{Koncepcja}
    // jaki system chcemy stworzyć
    // wektory, algorytmy klastrowania

    W tym rozdziale zostały opisane\ldots


    \section{Założenia i wymagania}
    W analizie wykorzystany zostanie głównie zbiór danych GDELT 2.0 od początku 2015 roku do kwietnia 2020.


    \section{Efekt końcowy}
    Planowanym efektem końcowym pracy będzie\ldots


    \chapter{Realizacja}


    \section{Wykorzystane narzędzia}

    \begin{enumerate}
        \item[•] język programowania Python~\cite{python}
        \item[•] biblioteka Pandas~\cite{pandas}
        \item[•] biblioteka GeoPandas~\cite{geopandas}
        \item[•] biblioteka Scikit-Learn~\cite{scikit}
        \item[•] środowisko programistyczne PyCharm~\cite{pycharm}
        \item[•] internetowe interaktywne środowisko obliczeniowe Notebook Jupyter~\cite{jupyter}
        \item[•] hurtownia danych Google BigQuery~\cite{bigquery}
    \end{enumerate}

    // dyskusja co można by wykorzystać zamiast tych narzędzi np dane lokalne zamiast bigQuery

    Przy tworzeniu projektu wykorzystany został język programowania Python.
    Jest to język wysokiego poziomu ogólnego przeznaczenia.
    Wraz z biblioteka Pandas jest często stosowany w zagadnieniach analizy danych oraz data miningu.
    Głównym narzędziem używanym do programowania było zintegrowane środowisko programistyczne PyCharm firmy JetBrains.
    Pozwala ono na wygodną edycję i analizę kodu źródłowego.
    Internetowe interaktywne środowisko obliczeniowe Notebook Jupyter pozwoliło na tworzenie dokumentów zawierających kod wraz z wizualizacjami.
    Wykorzystanie hurtowni danych Google BigQuery pozwoliło na szybką, skalowalną analizę dużego zbioru danych jakim jest GDELT.
    BigQuery jest oprogramowaniem bezserwerowym, które obsługuje zapytania w języku ANSI SQL.
    Podział krajów na klastry został wykonany przy użyciu Scikit-Learn.
    Do wizualizacji klastrów na mapie wykorzystano bibliotekę GeoPandas.


    \chapter{Wstępna analiza danych}
    W tej części pracy przeprowadzona zostanie wstępna analiza zagregowanych danych. W pierwszej kolejności przeanalizowane zostaną dane dotyczące Polski, co pozwoli na łatwiejsze wychwycenie związków między zarejestrowanymi wydarzeniami, a sytuacją w kraju.
    W dalszej kolejności przeprowadzona zostania analiza zbiorcza dla wszystkich krajów.


    \section{Popularność Polski w zbiorze danych GDELT}
    Jako pierwszą analizę wykonano badanie popularności Polski w zbiorze danych GDELT. Na wszystkich trzech wykresach obserwujemy znaczny wzrost liczby zdarzeń w 2015 roku. Może być to związane z uruchomieniem w GDELT automatycznego tłumaczenia artykułów i co za tym idzie zwiększeniem liczby źródeł danych.

    \paragraph{Polska jako Aktor 1}
    Wykres~\ref{fig:PLactor1} przedstawia popularność Polski jako aktora 1 skumulowaną dla poszczególnych lat.
    W roku 2016 obserwujemy szczyt popularności na poziomie około 150 tysięcy zdarzeń.
    \begin{figure}[!htp]
        \centering
        \includegraphics[width=\linewidth]{fig/PL/PLactor1.png}
        \caption{Liczba zdarzeń z Polską jako aktorem 1. (źródło: opracowanie własne)}
        \label{fig:PLactor1}
        \includegraphics[width=\linewidth]{fig/PL/PLactor2.png}
        \caption{Liczba zdarzeń z Polską jako aktorem 2. (źródło: opracowanie własne)}
        \label{fig:PLactor2}
    \end{figure}

    \paragraph{Polska jako Aktor 2}
    Wykres~\ref{fig:PLactor2} przedstawia popularność Polski jako aktora 2 skumulowaną dla poszczególnych lat. Kształt wykresu jest bardzo zbliżony do~\ref{fig:PLactor1} jednak szczyt popularności jest niższy - na poziomie około 130 tysięcy zdarzeń.

    \paragraph{Polska jako miejsce wydarzeń}
    Wykres~\ref{fig:PLlocation} przedstawia popularność Polski jako miejsca wydarzeń skumulowaną dla poszczególnych lat. Ponownie kształt wykresu jest zbliżony do~\ref{fig:PLactor1}. W tym przypadku szczyt popularności jest wyższy - na poziomie około 210 tysięcy zdarzeń.
    \begin{figure}[!htp]
        \centering
        \includegraphics[width=\linewidth]{fig/PL/PLlocation.png}
        \caption{Liczba zdarzeń z Polską jako lokacją. (źródło: opracowanie własne)}
        \label{fig:PLlocation}
    \end{figure}

    \subsection{Analiza zbiorcza od 2015 roku}
    Dane pochodzą z przedziału od stycznia 2015 do kwietnia 2020 roku.

    \paragraph{Liczba zdarzeń dla poszczególnych krajów}
    Wykres~\ref{fig:GLOBALactor1} przedstawia sumaryczną liczbę zdarzeń od 2015 roku, dla poszczególnych krajów, uszeregowaną malejąco.
    \begin{figure}[!htp]
        \centering
        \includegraphics[width=\linewidth]{fig/GLOBAL/Actor1.png}
        \caption{Liczba zdarzeń dla poszczególnych krajów. (źródło: opracowanie własne)}
        \label{fig:GLOBALactor1}
    \end{figure}
    Niekwestionowanym liderem pod względem liczby zdarzeń są Stany Zjednoczone. Dystansują one pozostałe kraje o prawie rząd wielkości.
    Kraje anglosaskie są szczególnie mocno reprezentowane.
    W czołówce pojawiają sie też kraje znaczące politycznie oraz silnie skonfliktowane.

    \paragraph{Liczba zdarzeń w czasie}
    Wykres~\ref{fig:GLOBALactor1inTime} przedstawia liczbę zdarzeń dla top 5 krajów skumulowaną dla poszczególnych lat.
    \begin{figure}[!htp]
        \centering
        \includegraphics[width=\linewidth]{fig/GLOBAL/Actor1inTIME.png}
        \caption{Liczba zdarzeń dla poszczególnych krajów w czasie. (źródło: opracowanie własne)}
        \label{fig:GLOBALactor1inTime}
    \end{figure}

    \paragraph{Popularność czterokodów zdarzeń}
    Wykres~\ref{fig:GLOBALQC} przedstawia sumaryczną liczbę zdarzeń od 2015 roku, dla poszczególnych czterokodów zdarzeń, uszeregowaną malejąco.
    \begin{figure}[!htp]
        \centering
        \includegraphics[width=\linewidth]{fig/GLOBAL/QC.png}
        \caption{Liczba zdarzeń dla poszczególnych kodów w czasie. (źródło: opracowanie własne)}
        \label{fig:GLOBALQC}
    \end{figure}

    \paragraph{Popularność czterokodów zdarzeń w czasie}
    Wykres~\ref{fig:GLOBALQCperc} przedstawia liczbę zdarzeń dla czterokodów zdarzeń skumulowaną dla poszczególnych lat.
    \begin{figure}[!htp]
        \centering
        \includegraphics[width=\linewidth]{fig/GLOBAL/QCperc.png}
        \caption{Procentowa liczba zdarzeń dla poszczególnych kodów w czasie. (źródło: opracowanie własne)}
        \label{fig:GLOBALQCperc}
    \end{figure}

    \paragraph{Popularność bazowych kodów zdarzeń}
    Wykres~\ref{fig:GLOBALEBC} przedstawia sumaryczną liczbę zdarzeń od 2015 roku, dla poszczególnych bazowych kodów zdarzeń, uszeregowaną malejąco.
    \begin{figure}[!htp]
        \centering
        \includegraphics[width=\linewidth]{fig/GLOBAL/EBC.png}
        \caption{Liczba zdarzeń dla poszczególnych kodów w czasie. (źródło: opracowanie własne)}
        \label{fig:GLOBALEBC}
    \end{figure}

    \paragraph{Popularność bazowych kodów zdarzeń w czasie}
    Wykres~\ref{fig:GLOBALEBCperc} przedstawia liczbę zdarzeń dla top 20 bazowych kodów zdarzeń skumulowaną dla poszczególnych lat.
    \begin{figure}[!htp]
        \centering
        \includegraphics[width=\linewidth]{fig/GLOBAL/EBCperc.png}
        \caption{Procentowa liczba zdarzeń dla poszczególnych kodów w czasie. (źródło: opracowanie własne)}
        \label{fig:GLOBALEBCperc}
    \end{figure}

    \paragraph{Popularność podstawowych kodów zdarzeń}
    Wykres~\ref{fig:GLOBALERC} przedstawia sumaryczną liczbę zdarzeń od 2015 roku, dla poszczególnych podstawowych kodów zdarzeń, uszeregowaną malejąco.
    \begin{figure}[!htp]
        \centering
        \includegraphics[width=\linewidth]{fig/GLOBAL/ERC.png}
        \caption{Liczba zdarzeń dla poszczególnych kodów w czasie. (źródło: opracowanie własne)}
        \label{fig:GLOBALERC}
    \end{figure}

    \paragraph{Popularność podstawowych kodów zdarzeń w czasie}
    Wykres~\ref{fig:GLOBALERCperc} przedstawia liczbę zdarzeń dla top 20 podstawowych kodów zdarzeń skumulowaną dla poszczególnych lat.
    \begin{figure}[!htp]
        \centering
        \includegraphics[width=\linewidth]{fig/GLOBAL/ERCperc.png}
        \caption{Procentowa liczba zdarzeń dla poszczególnych kodów w czasie. (źródło: opracowanie własne)}
        \label{fig:GLOBALERCperc}
    \end{figure}


    \section{Analiza danych dla wybranych krajów}

    \subsection{Polska}

    \paragraph{Kraj para do zdarzenia}


    Wykres~\ref{fig:PLpair} przedstawia liczbę zdarzeń dla Polski w których parą jest dany kraj.

    \begin{figure}[!htp]
        \centering
        \includegraphics[width=\linewidth]{fig/PL/PLactor2Pair.png}
        \caption{Liczba zdarzeń w których parą jest dany kraj. (źródło: opracowanie własne)}
        \label{fig:PLpair}
    \end{figure}


    Wykres~\ref{fig:PLpairPerc} przedstawia liczbę zdarzeń dla Polski w których parą jest dany kraj w czasie.
    \begin{figure}[!htp]
        \centering
        \includegraphics[width=\linewidth]{fig/PL/PLactor2PairPercinTIME.png}
        \caption{Procentowa liczba zdarzeń w których parą jest dany kraj w czasie. (źródło: opracowanie własne)}
        \label{fig:PLpairPerc}
    \end{figure}

    \paragraph{Podstawowy kod zdarzeń}

    Wykres~\ref{fig:PLPERC} przedstawia liczbę zdarzeń dla Polski dla poszczególnych kodów podstawowych.


    \begin{figure}[!htp]
        \centering
        \includegraphics[width=\linewidth]{fig/PL/PLERC.png}
        \caption{Liczba zdarzeń dla poszczególnych kodów podstawowych. (źródło: opracowanie własne)}
        \label{fig:PLPERC}
    \end{figure}

    Wykres~\ref{fig:PLPERCinTIME} przedstawia liczbę zdarzeń dla Polski dla poszczególnych kodów podstawowych w czasie.
    \begin{figure}[!htp]
        \centering
        \includegraphics[width=\linewidth]{fig/PL/PLERCinTIME.png}
        \caption{Liczba zdarzeń dla poszczególnych kodów podstawowych w czasie. (źródło: opracowanie własne)}
        \label{fig:PLPERCinTIME}
    \end{figure}

    Wykres~\ref{fig:PLPERCpercinTIME} przedstawia procentową liczbę zdarzeń dla Polski dla poszczególnych kodów podstawowych w czasie.
    \begin{figure}[!htp]
        \centering
        \includegraphics[width=\linewidth]{fig/PL/PLERCpercinTIME.png}
        \caption{Procentowa liczba zdarzeń dla poszczególnych kodów podstawowych w czasie. (źródło: opracowanie własne)}
        \label{fig:PLPERCpercinTIME}
    \end{figure}

    \paragraph{Podstawowe kody zdarzeń między Polską a wybranymi krajami}

    Wykres~\ref{fig:PLDEUERC} przedstawia liczbę zdarzeń z Niemcami dla poszczególnych kodów podstawowych w czasie.
    \begin{figure}[!htp]
        \centering
        \includegraphics[width=\linewidth]{fig/PL/POLDEUERCperc.png}
        \caption{Procentowa liczba zdarzeń z Niemcami dla poszczególnych kodów podstawowych w czasie. (źródło: opracowanie własne)}
        \label{fig:PLDEUERC}
    \end{figure}

    Wykres~\ref{fig:PLFRAERC} przedstawia liczbę zdarzeń z Francją dla poszczególnych kodów podstawowych w czasie.
    \begin{figure}[!htp]
        \centering
        \includegraphics[width=\linewidth]{fig/PL/POLFRAERCperc.png}
        \caption{Procentowa liczba zdarzeń z Francją dla poszczególnych kodów podstawowych w czasie. (źródło: opracowanie własne)}
        \label{fig:PLFRAERC}
    \end{figure}

    Wykres~\ref{fig:PLGBRERC} przedstawia liczbę zdarzeń z Wielką Brytanią dla poszczególnych kodów podstawowych w czasie.
    \begin{figure}[!htp]
        \centering
        \includegraphics[width=\linewidth]{fig/PL/POLGBRERCperc.png}
        \caption{Procentowa liczba zdarzeń z Wielka Brytanią dla poszczególnych kodów podstawowych w czasie. (źródło: opracowanie własne)}
        \label{fig:PLGBRERC}
    \end{figure}

    Wykres~\ref{fig:PLRUSERC} przedstawia liczbę zdarzeń z Rosją dla poszczególnych kodów podstawowych w czasie.
    \begin{figure}[!htp]
        \centering
        \includegraphics[width=\linewidth]{fig/PL/POLRUSERCperc.png}
        \caption{Procentowa liczba zdarzeń z Rosją dla poszczególnych kodów podstawowych w czasie. (źródło: opracowanie własne)}
        \label{fig:PLRUSERC}
    \end{figure}

    \subsection{Niemcy}

    \paragraph{Kraj para do zdarzenia}

    Wykres~\ref{fig:DEUpair} przedstawia liczbę zdarzeń dla Niemiec w których parą jest dany kraj.

    \begin{figure}[!htp]
        \centering
        \includegraphics[width=\linewidth]{fig/DEU/DEUactor2Pair.png}
        \caption{Liczba zdarzeń w których parą jest dany kraj. (źródło: opracowanie własne)}
        \label{fig:DEUpair}
    \end{figure}


    Wykres~\ref{fig:DEUpairPerc} przedstawia procentową liczbę zdarzeń dla Niemiec w których parą jest dany kraj w czasie.
    \begin{figure}[!htp]
        \centering
        \includegraphics[width=\linewidth]{fig/DEU/DEUactor2PairPercinTIME.png}
        \caption{Procentowa liczba zdarzeń w których parą jest dany kraj w czasie. (źródło: opracowanie własne)}
        \label{fig:DEUpairPerc}
    \end{figure}

    \paragraph{Podstawowy kod zdarzeń}

    Wykres~\ref{fig:DEUPERC} przedstawia liczbę zdarzeń dla Niemiec dla poszczególnych kodów podstawowych.


    \begin{figure}[!htp]
        \centering
        \includegraphics[width=\linewidth]{fig/DEU/DEUERC.png}
        \caption{Liczba zdarzeń dla poszczególnych kodów podstawowych. (źródło: opracowanie własne)}
        \label{fig:DEUPERC}
    \end{figure}

    Wykres~\ref{fig:DEUPERCinTIME} przedstawia liczbę zdarzeń dla Niemiec dla poszczególnych kodów podstawowych w czasie.
    \begin{figure}[!htp]
        \centering
        \includegraphics[width=\linewidth]{fig/DEU/DEUERCinTIME.png}
        \caption{Liczba zdarzeń dla poszczególnych kodów podstawowych w czasie. (źródło: opracowanie własne)}
        \label{fig:DEUPERCinTIME}
    \end{figure}

    Wykres~\ref{fig:DEUPERCpercinTIME} przedstawia procentową liczbę zdarzeń dla Niemiec dla poszczególnych kodów podstawowych w czasie.
    \begin{figure}[!htp]
        \centering
        \includegraphics[width=\linewidth]{fig/DEU/DEUERCpercinTIME.png}
        \caption{Procentowa liczba zdarzeń dla poszczególnych kodów podstawowych w czasie. (źródło: opracowanie własne)}
        \label{fig:DEUPERCpercinTIME}
    \end{figure}

    \paragraph{Podstawowe kody zdarzeń między Polską a wybranymi krajami}

    Wykres~\ref{fig:DEUPOLERC} przedstawia liczbę zdarzeń z Polską dla poszczególnych kodów podstawowych w czasie.
    \begin{figure}[!htp]
        \centering
        \includegraphics[width=\linewidth]{fig/DEU/DEUPOLERCperc.png}
        \caption{Procentowa liczba zdarzeń z Niemcami dla poszczególnych kodów podstawowych w czasie. (źródło: opracowanie własne)}
        \label{fig:DEUPOLERC}
    \end{figure}

    Wykres~\ref{fig:DEURUSERC} przedstawia liczbę zdarzeń z Rosją dla poszczególnych kodów podstawowych w czasie.
    \begin{figure}[!htp]
        \centering
        \includegraphics[width=\linewidth]{fig/DEU/DEURUSERCperc.png}
        \caption{Procentowa liczba zdarzeń z Francją dla poszczególnych kodów podstawowych w czasie. (źródło: opracowanie własne)}
        \label{fig:DEURUSERC}
    \end{figure}

    \subsection{Rosja}

    \paragraph{Kraj para do zdarzenia}

    Wykres~\ref{fig:RUSpair} przedstawia liczbę zdarzeń dla Niemiec w których parą jest dany kraj.

    \begin{figure}[!htp]
        \centering
        \includegraphics[width=\linewidth]{fig/RUS/RUSactor2Pair.png}
        \caption{Liczba zdarzeń w których parą jest dany kraj. (źródło: opracowanie własne)}
        \label{fig:RUSpair}
    \end{figure}


    Wykres~\ref{fig:RUSpairPerc} przedstawia procentową liczbę zdarzeń dla Niemiec w których parą jest dany kraj w czasie.
    \begin{figure}[!htp]
        \centering
        \includegraphics[width=\linewidth]{fig/RUS/RUSactor2PairPercinTIME.png}
        \caption{Procentowa liczba zdarzeń w których parą jest dany kraj w czasie. (źródło: opracowanie własne)}
        \label{fig:RUSpairPerc}
    \end{figure}

    \paragraph{Podstawowy kod zdarzeń}

    Wykres~\ref{fig:RUSPERC} przedstawia liczbę zdarzeń dla Rosji dla poszczególnych kodów podstawowych.

    \begin{figure}[!htp]
        \centering
        \includegraphics[width=\linewidth]{fig/RUS/RUSERC.png}
        \caption{Liczba zdarzeń dla poszczególnych kodów podstawowych. (źródło: opracowanie własne)}
        \label{fig:RUSPERC}
    \end{figure}

    Wykres~\ref{fig:RUSPERCinTIME} przedstawia liczbę zdarzeń dla Rosji dla poszczególnych kodów podstawowych w czasie.
    \begin{figure}[!htp]
        \centering
        \includegraphics[width=\linewidth]{fig/RUS/RUSERCinTIME.png}
        \caption{Liczba zdarzeń dla poszczególnych kodów podstawowych w czasie. (źródło: opracowanie własne)}
        \label{fig:RUSPERCinTIME}
    \end{figure}

    Wykres~\ref{fig:RUSPERCpercinTIME} przedstawia procentową liczbę zdarzeń dla Rosji dla poszczególnych kodów podstawowych w czasie.
    \begin{figure}[!htp]
        \centering
        \includegraphics[width=\linewidth]{fig/RUS/RUSERCpercinTIME.png}
        \caption{Procentowa liczba zdarzeń dla poszczególnych kodów podstawowych w czasie. (źródło: opracowanie własne)}
        \label{fig:RUSPERCpercinTIME}
    \end{figure}

    \paragraph{Podstawowe kody zdarzeń między Rosją a wybranymi krajami}

    Wykres~\ref{fig:RUSPOLERC} przedstawia liczbę zdarzeń z Polską dla poszczególnych kodów podstawowych w czasie.
    \begin{figure}[!htp]
        \centering
        \includegraphics[width=\linewidth]{fig/RUS/RUSPOLERCperc.png}
        \caption{Procentowa liczba zdarzeń z Niemcami dla poszczególnych kodów podstawowych w czasie. (źródło: opracowanie własne)}
        \label{fig:RUSPOLERC}
    \end{figure}

    Wykres~\ref{fig:RUSRUSERC} przedstawia liczbę zdarzeń z Niemcami dla poszczególnych kodów podstawowych w czasie.
    \begin{figure}[!htp]
        \centering
        \includegraphics[width=\linewidth]{fig/RUS/RUSDEUERCperc.png}
        \caption{Procentowa liczba zdarzeń z Francją dla poszczególnych kodów podstawowych w czasie. (źródło: opracowanie własne)}
        \label{fig:RUSRUSERC}
    \end{figure}

    \subsection{Stan zjednoczone}

    \paragraph{Kraj para do zdarzenia}

    Wykres~\ref{fig:USApair} przedstawia liczbę zdarzeń dla Stanów Zjednoczonych w których parą jest dany kraj.

    \begin{figure}[!htp]
        \centering
        \includegraphics[width=\linewidth]{fig/USA/USAactor2Pair.png}
        \caption{Liczba zdarzeń w których parą jest dany kraj. (źródło: opracowanie własne)}
        \label{fig:USApair}
    \end{figure}


    Wykres~\ref{fig:USApairPerc} przedstawia procentową liczbę zdarzeń dla Stanów Zjednoczonych w których parą jest dany kraj w czasie.
    \begin{figure}[!htp]
        \centering
        \includegraphics[width=\linewidth]{fig/USA/USAactor2PairPercinTIME.png}
        \caption{Procentowa liczba zdarzeń w których parą jest dany kraj w czasie. (źródło: opracowanie własne)}
        \label{fig:USApairPerc}
    \end{figure}

    \paragraph{Podstawowy kod zdarzeń}

    Wykres~\ref{fig:USAPERC} przedstawia liczbę zdarzeń dla Stanów Zjednoczonych dla poszczególnych kodów podstawowych.

    \begin{figure}[!htp]
        \centering
        \includegraphics[width=\linewidth]{fig/USA/USAERC.png}
        \caption{Liczba zdarzeń dla poszczególnych kodów podstawowych. (źródło: opracowanie własne)}
        \label{fig:USAPERC}
    \end{figure}

    Wykres~\ref{fig:USAPERCinTIME} przedstawia liczbę zdarzeń dla Stanów Zjednoczonych dla poszczególnych kodów podstawowych w czasie.
    \begin{figure}[!htp]
        \centering
        \includegraphics[width=\linewidth]{fig/USA/USAERCinTIME.png}
        \caption{Liczba zdarzeń dla poszczególnych kodów podstawowych w czasie. (źródło: opracowanie własne)}
        \label{fig:USAPERCinTIME}
    \end{figure}

    Wykres~\ref{fig:USAPERCpercinTIME} przedstawia procentową liczbę zdarzeń dla Stanów Zjednoczonych dla poszczególnych kodów podstawowych w czasie.
    \begin{figure}[!htp]
        \centering
        \includegraphics[width=\linewidth]{fig/USA/USAERCpercinTIME.png}
        \caption{Procentowa liczba zdarzeń dla poszczególnych kodów podstawowych w czasie. (źródło: opracowanie własne)}
        \label{fig:USAPERCpercinTIME}
    \end{figure}

    \paragraph{Podstawowe kody zdarzeń między Rosją a wybranymi krajami}

    Wykres~\ref{fig:USAPOLERC} przedstawia liczbę zdarzeń z Polską dla poszczególnych kodów podstawowych w czasie.
    \begin{figure}[!htp]
        \centering
        \includegraphics[width=\linewidth]{fig/USA/USAPOLERCperc.png}
        \caption{Procentowa liczba zdarzeń z Niemcami dla poszczególnych kodów podstawowych w czasie. (źródło: opracowanie własne)}
        \label{fig:USAPOLERC}
    \end{figure}

    Wykres~\ref{fig:USADEUERC} przedstawia liczbę zdarzeń z Niemcami dla poszczególnych kodów podstawowych w czasie.
    \begin{figure}[!htp]
        \centering
        \includegraphics[width=\linewidth]{fig/USA/USADEUERCperc.png}
        \caption{Procentowa liczba zdarzeń z Francją dla poszczególnych kodów podstawowych w czasie. (źródło: opracowanie własne)}
        \label{fig:USADEUERC}
    \end{figure}

    Wykres~\ref{fig:USARUSERC} przedstawia liczbę zdarzeń z Rosją dla poszczególnych kodów podstawowych w czasie.
    \begin{figure}[!htp]
        \centering
        \includegraphics[width=\linewidth]{fig/USA/USADEUERCperc.png}
        \caption{Procentowa liczba zdarzeń z Francją dla poszczególnych kodów podstawowych w czasie. (źródło: opracowanie własne)}
        \label{fig:USARUSERC}
    \end{figure}


    \section{Analiza siły powiązania}

    \subsection{Analiza siły powiązania miedzy wybranymi krajami}

    \paragraph{Polska}

    Wykres~\ref{fig:PLConnection} przedstawia siłę połączenia Polski z wybranymi krajami w czasie.


    \begin{figure}[!htp]
        \centering
        \includegraphics[width=\linewidth]{fig/PL/POLConnection.png}
        \caption{Siła połączenia Polski z wybranymi krajami w czasie. (źródło: opracowanie własne)}
        \label{fig:PLConnection}
    \end{figure}

    \paragraph{Rosja}

    Wykres~\ref{fig:RUSConnection} przedstawia siłę połączenia Rosji z wybranymi krajami w czasie.

    \begin{figure}[!htp]
        \centering
        \includegraphics[width=\linewidth]{fig/RUS/RUSConnection.png}
        \caption{Siła połączenia Rosji z wybranymi krajami w czasie. (źródło: opracowanie własne)}
        \label{fig:RUSConnection}
    \end{figure}

    \paragraph{Niemcy}

    Wykres~\ref{fig:DEUConnection} przedstawia siłę połączenia Niemiec z wybranymi krajami w czasie.

    \begin{figure}[!htp]
        \centering
        \includegraphics[width=\linewidth]{fig/DEU/DEUConnection.png}
        \caption{Siła połączenia Niemiec z wybranymi krajami w czasie. (źródło: opracowanie własne)}
        \label{fig:DEUConnection}
    \end{figure}

    \paragraph{Stany Zjednoczone}

    Wykres~\ref{fig:PLConnection} przedstawia siłę połączenia Stanów Zjednoczonych z wybranymi krajami w czasie.

    \begin{figure}[!htp]
        \centering
        \includegraphics[width=\linewidth]{fig/USA/USAConnection.png}
        \caption{Siła połączenia Stanów Zjednoczonych z wybranymi krajami w czasie. (źródło: opracowanie własne)}
        \label{fig:USAConnection}
    \end{figure}

    \subsection{Analiza symetryczności siły powiązania}

    \paragraph{Polska - Niemcy - Polska}

    Wykres~\ref{fig:POL-DEU-POL} przedstawia symetryczność siły połączenia Polski i Niemiec w czasie.


    \begin{figure}[!htp]
        \centering
        \includegraphics[width=\linewidth]{fig/ConnectionSymmetry/POL-DEU-POL.png}
        \caption{Symetryczność siły połączenia Polski i Niemiec w czasie. (źródło: opracowanie własne)}
        \label{fig:POL-DEU-POL}
    \end{figure}

    \paragraph{Polska - Rosja - Polska}

    Wykres~\ref{fig:POL-RUS-POL} przedstawia symetryczność siły połączenia Polski i Rosji w czasie.


    \begin{figure}[!htp]
        \centering
        \includegraphics[width=\linewidth]{fig/ConnectionSymmetry/POL-RUS-POL.png}
        \caption{Symetryczność siły połączenia Polski i Rosji w czasie. (źródło: opracowanie własne)}
        \label{fig:POL-RUS-POL}
    \end{figure}

    \paragraph{Polska - Stany Zjednoczone - Polska}

    Wykres~\ref{fig:POL-USA-POL} przedstawia symetryczność siły połączenia Polski i Stanów Zjednoczonych w czasie.


    \begin{figure}[!htp]
        \centering
        \includegraphics[width=\linewidth]{fig/ConnectionSymmetry/POL-USA-POL.png}
        \caption{Symetryczność siły połączenia Polski i Stanów Zjednoczonych w czasie. (źródło: opracowanie własne)}
        \label{fig:POL-USA-POL}
    \end{figure}


    \section{Analiza kodu podstawowego Fight}

    Wykres~\ref{fig:Fight} przedstawia procentową liczbę państw z jakimi dany kraj ma zdarzenia fight w czasie.

    \begin{figure}[!htp]
        \centering
        \includegraphics[width=\linewidth]{fig/ERC/Fight.png}
        \caption{Procentowa liczba państw z jakimi dany kraj ma zdarzenia fight w czasie. (źródło: opracowanie własne)}
        \label{fig:Fight}
    \end{figure}


    \chapter[Klasteryzacja K-Means]{Klasteryzacja z wykorzystaniem K-Means}

    Do przeprowadzenia grupowania krajów wykorzystany został wektor składający się z:
    \begin{enumerate}
        \item[•] liczby zdarzeń dla których kraj jest aktorem 1 (events),
        \item[•] liczby wzmianek (numMentions) - całkowitej liczba wzmianek o tym wydarzeniu, we wszystkich dokumentach źródłowych podczas 15-minutowej aktualizacji, w której zostało po raz pierwszy zauważone,
        \item[•] stosunku liczby zdarzeń material conflict do material cooperation z quad description (materialConfCoop) - stosunek czterokodów z podstawowej klasyfikacji
        \item[•] stosunku liczby zdarzeń verbal conflict do verbal cooperation z quad description (verbalConfCoop) - stosunek czterokodów z podstawowej klasyfikacji
        \item[•] średniego średniego tonu (avgAvgTone) - średni „ton” wszystkich dokumentów zawierających jedną lub więcej wzmianek o tym wydarzeniu, podczas 15-minutowej aktualizacji, w której zostało ono po raz pierwszy zauważone. Waha się od -100 (skrajnie ujemny) do +100 (skrajnie dodatni).
        \item[•] średniej miary Goldsteina (avgGoldstein) - skala Goldsteina przypisuje wynik liczbowy od -10 do +10, wychwytując teoretyczny potencjalny wpływ, jaki rodzaj zdarzenia będzie miał na stabilność kraju,
        \item[•] liczby zdarzeń Fight (fightCount)
        \item[•] liczby zdarzeń Express intent to cooperate (expressCount)
    \end{enumerate}
    Dane wykorzystane w tym doświadczeniu pochodzą ze stycznia 2020 roku.
    Przed dokonaniem klasteryzacji odrzucone zostały wydarzenia z kodami krajów cameo niezgodnymi z kodami ISO 3166-1 alfa-3~\cite{iso_alfa3}.
    Klasteryzacja została przeprowadzona dwukrotnie, za drugim razem na danych ustandaryzowanych przy pomocy modułu StandardScaler~\cite{standardScaler}.


    \section{Dane niestandaryzowane}
    W pierwszej kolejności zostaną przedstawione wyniki grupowania na danych niestandaryzowanych.

    Wykres~\ref{fig:clust10} przedstawia mapę z naniesionymi wynikami klasteryzacji. Każda grupa krajów otrzymała inny kolor.

    \begin{figure}[!htp]
        \centering
        \includegraphics[width=\linewidth]{fig/CLUST/10clusterMap.png}
        \caption{Mapa z wynikami klasteryzacji. (źródło: opracowanie własne)}
        \label{fig:clust10}
    \end{figure}

    \subsection{Wyniki klasteryzacji w postaci tabelarycznej}
    W tabelach od~\ref{tab:cl0} do~\ref{tab:cl9} przedstawione zostały wyniki grupowania.
    Dodatkowo zawarto informacje o liczbie ludności i PKB na osobę pochodzące z biblioteki GeoPandas~\cite{geopandas}.

    \begin{table}[!htp]
        \centering
        \includegraphics[width=\linewidth]{tables/CLUST/clust0kmeans.png}
        \caption{Klaster 0. (źródło: opracowanie własne)}
        \label{tab:cl0}
    \end{table}

    \begin{table}[!htp]
        \centering
        \includegraphics[width=\linewidth]{tables/CLUST/clust1kmeans.png}
        \caption{Klaster 1. (źródło: opracowanie własne)}
        \label{tab:cl1}
    \end{table}

    \begin{table}[!htp]
        \centering
        \includegraphics[width=\linewidth]{tables/CLUST/clust2kmeans.png}
        \caption{Klaster 2. (źródło: opracowanie własne)}
        \label{tab:cl2}
    \end{table}

    \begin{table}[!htp]
        \centering
        \includegraphics[width=\linewidth]{tables/CLUST/clust3kmeans.png}
        \caption{Klaster 3. (źródło: opracowanie własne)}
        \label{tab:cl3}
    \end{table}

    \begin{table}[!htp]
        \centering
        \includegraphics[width=\linewidth]{tables/CLUST/clust4kmeans.png}
        \caption{Klaster 4. (źródło: opracowanie własne)}
        \label{tab:cl4}
    \end{table}

    \begin{table}[!htp]
        \centering
        \includegraphics[width=\linewidth]{tables/CLUST/clust5kmeans.png}
        \caption{Klaster 5. (źródło: opracowanie własne)}
        \label{tab:cl5}
    \end{table}

    \begin{table}[!htp]
        \centering
        \includegraphics[width=\linewidth]{tables/CLUST/clust6kmeans.png}
        \caption{Klaster 6. (źródło: opracowanie własne)}
        \label{tab:cl6}
    \end{table}

    \begin{table}[!htp]
        \centering
        \includegraphics[width=\linewidth]{tables/CLUST/clust7kmeans.png}
        \caption{Klaster 7. (źródło: opracowanie własne)}
        \label{tab:cl7}
    \end{table}

    \begin{table}[!htp]
        \centering
        \includegraphics[width=\linewidth]{tables/CLUST/clust8kmeans.png}
        \caption{Klaster 8. (źródło: opracowanie własne)}
        \label{tab:cl8}
    \end{table}

    \begin{table}[!htp]
        \centering
        \includegraphics[width=\linewidth]{tables/CLUST/clust9kmeans.png}
        \caption{Klaster 9. (źródło: opracowanie własne)}
        \label{tab:cl9}
    \end{table}

    Dla danych niestandaryzowanych Polska trafiła do klastra~\ref{tab:cl2} między innymi z Egiptem, Grecją, Irlandia oraz Brazylią.
    Wyróżnia się mniejsza grupa~\ref{tab:cl6} do której trafiły Chiny, Rosja, Wielka Brytania oraz Iran.


    \section{Dane ustandaryzowane}
    Standrd Scaler standaryzuje cechy poprzez usunięcie średniej i skalowanie do wariancji jednostkowej.
    Standardowy wynik próbki x jest obliczany jako:
    z = (x - u) / s
    gdzie u jest średnią próbek, a s jest standardowym odchyleniem próbek.

    Wykres~\ref{fig:clust10std} przedstawia mapę z naniesionymi wynikami klasteryzacji ustandaryzowanych próbek.
    Każda grupa krajów otrzymała inny kolor.

    \begin{figure}[!htp]
        \centering
        \includegraphics[width=\linewidth]{fig/CLUST/10clusterMap_std.png}
        \caption{Mapa z wynikami klasteryzacji. (źródło: opracowanie własne)}
        \label{fig:clust10std}
    \end{figure}

    Aby ułatwić interpretację wyników klasteryzacji poniżej dołączona została mapa~\ref{fig:clustPop} z naniesioną populacją oraz mapa~\ref{fig:clustGDP} z naniesionym PKB na osobę poszczególnych krajów, a także ich odpowiedniki ze skalą logarytmiczną~\ref{fig:clustPop_log} oraz~\ref{fig:clustGDP_log}.

    \begin{figure}[!htp]
        \centering
        \includegraphics[width=\linewidth]{fig/CLUST/population.png}
        \caption{Mapa z populacją krajów. (źródło: opracowanie własne)}
        \label{fig:clustPop}
    \end{figure}

    \begin{figure}[!htp]
        \centering
        \includegraphics[width=\linewidth]{fig/CLUST/gdp.png}
        \caption{Mapa z PKP na osobę. (źródło: opracowanie własne)}
        \label{fig:clustGDP}
    \end{figure}

    \begin{figure}[!htp]
        \centering
        \includegraphics[width=\linewidth]{fig/CLUST/population_log.png}
        \caption{Mapa z populacją krajów - skala logarytmiczna. (źródło: opracowanie własne)}
        \label{fig:clustPop_log}
    \end{figure}

    \begin{figure}[!htp]
        \centering
        \includegraphics[width=\linewidth]{fig/CLUST/gdp_log.png}
        \caption{Mapa z PKB na osobę - skala logarytmiczna. (źródło: opracowanie własne)}
        \label{fig:clustGDP_log}
    \end{figure}

    \begin{figure}[!htp]
        \centering
        \includegraphics[width=\linewidth]{fig/CLUST/gdp2015.png}
        \caption{Mapa z PKB - skala logarytmiczna. (źródło: opracowanie własne)}
        \label{fig:clustGDP2015_log}
    \end{figure}

    \begin{figure}[!htp]
        \centering
        \includegraphics[width=\linewidth]{fig/CLUST/health2015.png}
        \caption{Mapa z wydatkami na zdrowie - skala logarytmiczna. (źródło: opracowanie własne)}
        \label{fig:clustHealth2015_log}
    \end{figure}

    \begin{figure}[!htp]
        \centering
        \includegraphics[width=\linewidth]{fig/CLUST/military2015.png}
        \caption{Mapa z wydatkami na zbrojenia - skala logarytmiczna. (źródło: opracowanie własne)}
        \label{fig:clustMilitary2015_log}
    \end{figure}

    \begin{figure}[!htp]
        \centering
        \includegraphics[width=\linewidth]{fig/CLUST/import2015.png}
        \caption{Mapa z importem - skala logarytmiczna. (źródło: opracowanie własne)}
        \label{fig:clustImport2015_log}
    \end{figure}

    \begin{figure}[!htp]
        \centering
        \includegraphics[width=\linewidth]{fig/CLUST/export2015.png}
        \caption{Mapa z eksportem - skala logarytmiczna. (źródło: opracowanie własne)}
        \label{fig:clustExport2015_log}
    \end{figure}

    \subsection{Wyniki klasteryzacji - wykresy gęstości - dane ustandaryzowane}
    Na wykresach od~\ref{fig:density_events} do~\ref{fig:density_expresscount} przedstawione zostały wykresy gęstości miar wykorzystanych przy grupowaniu w poszczególnych klastrach.

    \begin{figure}[!htp]
        \centering
        \includegraphics[width=\linewidth]{fig/CLUST/density_Events.png}
        \caption{Gęstość miary Events. (źródło: opracowanie własne)}
        \label{fig:density_events}
    \end{figure}

    Na wykresie~\ref{fig:density_events} obserwujemy słabą separację klastrów.
    Wyróżniają się klastry 7 oraz 8 których gęstości są spłaszczone i wydłużone.

    \begin{figure}[!htp]
        \centering
        \includegraphics[width=\linewidth]{fig/CLUST/density_sumNumMentions.png}
        \caption{Gęstość miary sumNumMentions. (źródło: opracowanie własne)}
        \label{fig:density_sumnummentions}
    \end{figure}

    Na wykresie~\ref{fig:density_sumnummentions}, podobnie jak na poprzednim, obserwujemy słabą separację klastrów.
    Ponownie wyróżniają się klastry 7 oraz 8.

    \begin{figure}[!htp]
        \centering
        \includegraphics[width=\linewidth]{fig/CLUST/density_materialConfCoop.png}
        \caption{Gęstość miary materialConfCoop. (źródło: opracowanie własne)}
        \label{fig:density_materialconfcoop}
    \end{figure}

    Na wykresie~\ref{fig:density_materialconfcoop} obserwujemy dobrą separację klastrów 1, 3, 4, 5 oraz 8.

    \begin{figure}[!htp]
        \centering
        \includegraphics[width=\linewidth]{fig/CLUST/density_verbalConfCoop.png}
        \caption{Gęstość miary verbalConfCoop. (źródło: opracowanie własne)}
        \label{fig:density_verbalconfcoop}
    \end{figure}

    Na wykresie~\ref{fig:density_verbalconfcoop} zauważamy podobieństwo klastrów 1 i 9 oraz 3 i 7.

    \begin{figure}[!htp]
        \centering
        \includegraphics[width=\linewidth]{fig/CLUST/density_avgAvgTone.png}
        \caption{Gęstość miary avgAvgTone. (źródło: opracowanie własne)}
        \label{fig:density_avgavgtone}
    \end{figure}

    Na wykresie~\ref{fig:density_avgavgtone} wyróżnia się klaster 8 (największa gęstość) oraz klastry 7 i 8.


    \begin{figure}[!htp]
        \centering
        \includegraphics[width=\linewidth]{fig/CLUST/density_avgGoldstein.png}
        \caption{Gęstość miary avgGoldstein. (źródło: opracowanie własne)}
        \label{fig:density_avggoldstein}
    \end{figure}

    Na wykresie~\ref{fig:density_avggoldstein} najbardziej wyróżnia się klaster 8.
    Pozostałe (z wyjątkiem 2, 7 oraz 9) wyraźnie się oddzielają.

    \begin{figure}[!htp]
        \centering
        \includegraphics[width=\linewidth]{fig/CLUST/density_fightCount.png}
        \caption{Gęstość miary fightCount. (źródło: opracowanie własne)}
        \label{tab:density_fightCount}
    \end{figure}

    Na wykresie~\ref{tab:density_fightCount} większość klastrów osiąga maksimum gęstości w pobliżu wartości 0.

    \begin{figure}[!htp]
        \centering
        \includegraphics[width=\linewidth]{fig/CLUST/density_expressCount.png}
        \caption{Gęstość miary expressCount. (źródło: opracowanie własne)}
        \label{fig:density_expresscount}
    \end{figure}

    Na wykresie~\ref{fig:density_expresscount}, podobnie jak na poprzednim, obserwujemy skupienie w pobliżu zera.
    Wyróżniają się klastry 7 i 8, które są spłaszczone i wydłużone.

    \subsection{Wyniki klasteryzacji w postaci tabelarycznej - dane ustandaryzowane}
    W tabelach od~\ref{tab:cl0std} do~\ref{tab:cl9std} przedstawione zostały wyniki grupowania ustandaryzowanych próbek.
    Dla ułatwienia interpretacji wyników grupowania razem z nazwami krajów zostały dodane informacje o PKB, wydatkach na zdrowie, zbrojenia, edukację, import oraz eksport (jako procent PKB).
    Dodatkowe dane pochodzą z bazy Banku Światowego~\cite{worldbank}, są na licencji CC BY-4.0~\cite{wblicense}, przedstawiają sytuację w 2015 roku.
    \begin{enumerate}
        \item[GDP] PKB to suma wartości dodanej brutto, w dolarach, wszystkich producentów będących rezydentami w gospodarce powiększona o wszelkie podatki od produktów i pomniejszona o wszelkie dotacje nieuwzględnione w wartości produktów.
        \item[Education] Wydatki sektora instytucji rządowych i samorządowych na edukację są wyrażone jako procent PKB.
        \item[Military] Dane dotyczące wydatków wojskowych z Międzynarodowego Instytutu Badań nad Pokojem w Sztokholmie pochodzą z definicji NATO, która obejmuje wszystkie bieżące i kapitałowe wydatki na siły zbrojne, w tym siły pokojowe.
        Wartości mniejsze niż 0.01\% zostały odrzucone w celu dostosowania skali logarytmicznej.
        \item[Health] Poziom bieżących wydatków na zdrowie wyrażony jako procent PKB. Szacunki bieżących wydatków na zdrowie obejmują towary i usługi zdrowotne konsumowane w każdym roku.
        \item[Import] Import towarów i usług reprezentuje wartość wszystkich towarów i innych usług rynkowych otrzymanych z reszty świata.
        Obejmuje wartość towarów, frachtu, ubezpieczenia, transportu, podróży, tantiem, opłat licencyjnych i innych usług, takich jak usługi komunikacyjne, budowlane, finansowe, informacyjne, biznesowe, osobiste i rządowe.
        Nie obejmuje kosztów związanych z zatrudnieniem i dochodów z inwestycji oraz płatności transferowych.
        \item[Export] Eksport towarów i usług reprezentuje wartość wszystkich towarów i innych usług rynkowych dostarczanych do reszty świata.
    \end{enumerate}
    Dodatkowo dokonano kolorowania pól z informacjami o krajach.
    Poszczególne miary zostały podzielone na 5 równych przedziałów wg skali logarytmicznej.
    Kolory, od przedziału z najmniejszymi wartościami do tego z największymi, to: czerwony (bardzo niskie), pomarańczowy (niskie), żółty (średnie), jasnozielony (wysokie), zielony (bardzo wysokie).
    W tabelach od~\ref{tab:cl1stdcount} do~\ref{tab:cl9stdcount} została przedstawiona procentowa ilość państw w poszczególnych przedziałach w obrębie klastrów.
    W tabelach od~\ref{tab:cl0std_desc} do~\ref{tab:cl9std_desc} pokazane zostały parametry poszczególnych klastrów: średnia, mediana, odchylenie standardowe, minimum, maksimum.


    \begin{table}[!htp]
        \centering
        \includegraphics[width=\linewidth]{tables/CLUST/cluster0stdkmeans.png}
        \caption{Klaster 0 - dane standaryzowane. (źródło: opracowanie własne)}
        \label{tab:cl0std}
    \end{table}

    \begin{table}[!htp]
        \centering
        \includegraphics[width=\linewidth]{tables/CLUST/cluster0stdkmeanscount.png}
        \caption{Klaster 0 - ilość państw w poszczególnych przedziałach. (źródło: opracowanie własne)}
        \label{tab:cl0stdcount}
    \end{table}

    Klaster 0 w tabeli~\ref{tab:cl0std} zawiera 30 krajów.
    Przykłady ważniejszych państw w tym klastrze to: Estonia, Islandia, Luksemburg, Portugalia, Zjednoczone Emiraty Arabskie, Wietnam.
    W tabeli~\ref{tab:cl0stdcount} obserwujemy, że w klastrze 0 przeważają (2/3) kraje z niskimi i średnimi wydatkami na opiekę zdrowotną (pomiędzy 3, a 7.9\% PKB).

    \begin{table}[!htp]
        \centering
        \includegraphics[width=\linewidth]{tables/CLUST/desc/clust0std_desc.png}
        \caption{Parametry klastra 0 - dane standaryzowane. (źródło: opracowanie własne)}
        \label{tab:cl0std_desc}
    \end{table}

    \begin{table}[!htp]
        \centering
        \includegraphics[width=\linewidth]{tables/CLUST/cluster1stdkmeans.png}
        \caption{Klaster 1 - dane standaryzowane. (źródło: opracowanie własne)}
        \label{tab:cl1std}
    \end{table}

    \begin{table}[!htp]
        \centering
        \includegraphics[width=\linewidth]{tables/CLUST/cluster1stdkmeanscount.png}
        \caption{Klaster 1 - ilość państw w poszczególnych przedziałach. (źródło: opracowanie własne)}
        \label{tab:cl1stdcount}
    \end{table}

    Klaster 1 w tabeli~\ref{tab:cl1stdcount} zawiera 3 kraje - Indie, Liban oraz Palestynę.
    W tabeli~\ref{tab:cl1stdcount} obserwujemy, że wszystkie państwa w klastrze 1 charakteryzują się średnim eksportem (pomiędzy 2.5\%, a 5.1\% PKB).

    \begin{table}[!htp]
        \centering
        \caption{Parametry klastra 1 - dane standaryzowane. (źródło: opracowanie własne)}
        \label{tab:cl1std_desc}
        \includegraphics[width=\linewidth]{tables/CLUST/desc/clust1std_desc.png}
    \end{table}

    \begin{table}[!htp]
        \centering
        \includegraphics[width=\linewidth]{tables/CLUST/cluster2stdkmeans.png}
        \caption{Klaster 2 - dane standaryzowane. (źródło: opracowanie własne)}
        \label{tab:cl2std}
    \end{table}

    \begin{table}[!htp]
        \centering
        \includegraphics[width=\linewidth]{tables/CLUST/cluster2stdkmeanscount.png}
        \caption{Klaster 2 - ilość państw w poszczególnych przedziałach. (źródło: opracowanie własne)}
        \label{tab:cl2stdcount}
    \end{table}

    Klaster 2 w tabeli~\ref{tab:cl2stdcount} zawiera 3 kraje - Botswanę, Gwineę Bissau oraz .
    W tabeli~\ref{tab:cl2stdcount} obserwujemy, że wszystkie państwa w klastrze 2 charakteryzują się niskim eksportem (pomiędzy 1.2\%, a 2.5\% PKB).

    \begin{table}[!htp]
        \centering
        \includegraphics[width=\linewidth]{tables/CLUST/desc/clust2std_desc.png}
        \caption{Parametry klastra 2 - dane standaryzowane. (źródło: opracowanie własne)}
        \label{tab:cl2std_desc}
    \end{table}

    \begin{table}[!htp]
        \centering
        \includegraphics[width=\linewidth]{tables/CLUST/cluster3stdkmeans.png}
        \caption{Klaster 3 - dane standaryzowane. (źródło: opracowanie własne)}
        \label{tab:cl3std}
    \end{table}

    \begin{table}[!htp]
        \centering
        \includegraphics[width=\linewidth]{tables/CLUST/desc/clust3std_desc.png}
        \caption{Parametry klastra 3 - dane standaryzowane. (źródło: opracowanie własne)}
        \label{tab:cl3std_desc}
    \end{table}

    \begin{table}[!htp]
        \centering
        \includegraphics[width=\linewidth]{tables/CLUST/cluster3stdkmeanscount.png}
        \caption{Klaster 3 - ilość państw w poszczególnych przedziałach. (źródło: opracowanie własne)}
        \label{tab:cl3stdcount}
    \end{table}

    Klaster 3 w tabeli~\ref{tab:cl3stdcount} zwiera 10 krajów.
    Przykłady ważniejszych państw w tym klastrze to: Chiny, Francja, Niemcy, Rosja, Wielka Brytania.
    W tabeli~\ref{tab:cl3stdcount} obserwujemy, że w klastrze 3 przeważają kraje z wysokim oraz bardzo wysokim PKB (pomiędzy 3.23e+11\$, a 1.82e+13\$).

    \begin{table}[!htp]
        \centering
        \includegraphics[width=\linewidth]{tables/CLUST/cluster4stdkmeans.png}
        \caption{Klaster 4 - dane standaryzowane. (źródło: opracowanie własne)}
        \label{tab:cl4std}
    \end{table}

    \begin{table}[!htp]
        \centering
        \includegraphics[width=\linewidth]{tables/CLUST/cluster4stdkmeanscount.png}
        \caption{Klaster 4 - ilość państw w poszczególnych przedziałach. (źródło: opracowanie własne)}
        \label{tab:cl4stdcount}
    \end{table}

    Klaster 4 w tabeli~\ref{tab:cl4stdcount} zawiera 8 krajów.
    Przykłady ważniejszych państw w tym klastrze to: Kenia, Somalia, Syria.
    W tabeli~\ref{tab:cl4stdcount} obserwujemy, że w klastrze 4 przeważają kraje z niskim PKB (pomiędzy 5.71e+09\$, a 4.29e+10\$).

    \begin{table}[!htp]
        \centering
        \includegraphics[width=\linewidth]{tables/CLUST/desc/clust4std_desc.png}
        \caption{Parametry klastra 4 - dane standaryzowane. (źródło: opracowanie własne)}
        \label{tab:cl4std_desc}
    \end{table}

    \begin{table}[!htp]
        \centering
        \includegraphics[width=\linewidth]{tables/CLUST/cluster5stdkmeans.png}
        \caption{Klaster 5 - dane standaryzowane. (źródło: opracowanie własne)}
        \label{tab:cl5std}
    \end{table}

    \begin{table}[!htp]
        \centering
        \includegraphics[width=\linewidth]{tables/CLUST/cluster5stdkmeanscount.png}
        \caption{Klaster 5 - ilość państw w poszczególnych przedziałach. (źródło: opracowanie własne)}
        \label{tab:cl5stdcount}
    \end{table}

    Klaster 5 w tabeli~\ref{tab:cl5stdcount} zawiera 21 krajów.
    Przykłady ważniejszych państw w tym klastrze to: Chile, Nigeria, Pakistan, Polska, Szwecja.
    W tabeli~\ref{tab:cl5stdcount} obserwujemy, że w klastrze 5 przeważają kraje ze średnimi wydatkami na opiekę zdrowotną (pomiędzy 4.9\%, a 7.9\% PKB).

    \begin{table}[!htp]
        \centering
        \includegraphics[width=\linewidth]{tables/CLUST/desc/clust5std_desc.png}
        \caption{Parametry klastra 5 - dane standaryzowane. (źródło: opracowanie własne)}
        \label{tab:cl5std_desc}
    \end{table}

    \begin{table}[!htp]
        \centering
        \includegraphics[width=\linewidth]{tables/CLUST/cluster6stdkmeans.png}
        \caption{Klaster 6 - dane standaryzowane. (źródło: opracowanie własne)}
        \label{tab:cl6std}
    \end{table}

    \begin{table}[!htp]
        \centering
        \includegraphics[width=\linewidth]{tables/CLUST/cluster6stdkmeanscount.png}
        \caption{Klaster 6 - ilość państw w poszczególnych przedziałach. (źródło: opracowanie własne)}
        \label{tab:cl6stdcount}
    \end{table}

    Klaster 6 w tabeli~\ref{tab:cl6stdcount} zawiera 49 krajów.
    Przykłady ważniejszych państw w tym klastrze to: Białoruś, Bułgaria, Czechy, Finlandia, Węgry, Norwegia.
    W tabeli~\ref{tab:cl6stdcount} obserwujemy, że w klastrze 6 przeważają kraje z wysokimi wydatkami na edukację (pomiędzy 4\%, a 5.5\% PKB).

    \begin{table}[!htp]
        \centering
        \includegraphics[width=\linewidth]{tables/CLUST/desc/clust6std_desc.png}
        \caption{Parametry klastra 6 - dane standaryzowane. (źródło: opracowanie własne)}
        \label{tab:cl6std_desc}
    \end{table}

    \begin{table}[!htp]
        \centering
        \includegraphics[width=\linewidth]{tables/CLUST/cluster7stdkmeans.png}
        \caption{Klaster 7 - dane standaryzowane. (źródło: opracowanie własne)}
        \label{tab:cl7std}
    \end{table}

    \begin{table}[!htp]
        \centering
        \includegraphics[width=\linewidth]{tables/CLUST/cluster7stdkmeanscount.png}
        \caption{Klaster 7 - ilość państw w poszczególnych przedziałach. (źródło: opracowanie własne)}
        \label{tab:cl7stdcount}
    \end{table}

    Klaster 7 w tabeli~\ref{tab:cl7stdcount} zawiera 2 kraje - Iran i Irak.
    W tabeli~\ref{tab:cl7stdcount} obserwujemy, że w klastrze 7 żaden z parametrów nie wyróżnia się.

    \begin{table}[!htp]
        \centering
        \includegraphics[width=\linewidth]{tables/CLUST/desc/clust7std_desc.png}
        \caption{Parametry klastra 7 - dane standaryzowane}
        \label{tab:cl7std_desc}
    \end{table}

    \begin{table}[!htp]
        \centering
        \includegraphics[width=\linewidth]{tables/CLUST/cluster8stdkmeans.png}
        \caption{Klaster 8 - dane standaryzowane. (źródło: opracowanie własne)}
        \label{tab:cl8std}
    \end{table}

    \begin{table}[!htp]
        \centering
        \includegraphics[width=\linewidth]{tables/CLUST/cluster8stdkmeanscount.png}
        \caption{Klaster 8 - ilość państw w poszczególnych przedziałach. (źródło: opracowanie własne)}
        \label{tab:cl8stdcount}
    \end{table}

    Klaster 8 w tabeli~\ref{tab:cl8stdcount} posiada tylko jedno państwo - Stany Zjednoczone Ameryki.
    W tabeli~\ref{tab:cl8stdcount} obserwujemy, że klaster 8 wyróżnia sie bardzo wysokim PKB (2.42e+12\$, a 1.82e+13\$).

    \begin{table}[!htp]
        \centering
        \includegraphics[width=\linewidth]{tables/CLUST/desc/clust8std_desc.png}
        \caption{Parametry klastra 8 - dane standaryzowane. (źródło: opracowanie własne)}
        \label{tab:cl8std_desc}
    \end{table}

    \begin{table}[!htp]
        \centering
        \includegraphics[width=\linewidth]{tables/CLUST/cluster9stdkmeans.png}
        \caption{Klaster 9 - dane standaryzowane. (źródło: opracowanie własne)}
        \label{tab:cl9std}
    \end{table}

    \begin{table}[!htp]
        \centering
        \includegraphics[width=\linewidth]{tables/CLUST/cluster9stdkmeanscount.png}
        \caption{Klaster 9 - ilość państw w poszczególnych przedziałach. (źródło: opracowanie własne)}
        \label{tab:cl9stdcount}
    \end{table}

    Klaster 9 w tabeli~\ref{tab:cl9stdcount} zawiera 31 krajów.
    Przykłady ważniejszych państw w tym klastrze to: Gwatemala, Honduras, Kuwejt, Meksyk, Słowacja.
    W tabeli~\ref{tab:cl9stdcount} obserwujemy, że klaster 9 wyróżnia sie niskimi wydatkami na zbrojenia (pomiędzy 0.7\%, a 1.48\% PKB).

    \begin{table}[!htp]
        \centering
        \includegraphics[width=\linewidth]{tables/CLUST/desc/clust9std_desc.png}
        \caption{Parametry klastra 9 - dane standaryzowane. (źródło: opracowanie własne)}
        \label{tab:cl9std_desc}
    \end{table}

    Tabele~\ref{tab:cl_mean_summ} oraz~\ref{tab:cl_median_summ} zawierają podsumowanie parametrów poszczególnych klastrów.

    \begin{table}[!htp]
        \centering
        \includegraphics[width=\linewidth]{tables/CLUST/desc/cluster_mean_summary.png}
        \caption{Średnie wartości parametrów w klastrach. (źródło: opracowanie własne)}
        \label{tab:cl_mean_summ}
    \end{table}

    \begin{table}[!htp]
        \centering
        \includegraphics[width=\linewidth]{tables/CLUST/desc/cluster_median_summary.png}
        \caption{Mediany wartości parametrów w klastrach. (źródło: opracowanie własne)}
        \label{tab:cl_median_summ}
    \end{table}


    Zarówno dla danych nie poddanych oraz poddanych standaryzacji Stany Zjednoczono otrzymały własny klaster -~\ref{tab:cl7} oraz ~\ref{tab:cl0std}.
    Jest to spowodowane znaczącą przewagą liczby zdarzeń w porównaniu z innymi krajami.

    Dla danych standaryzowanych Polska jest jedynym krajem europejskim w klastrze~\ref{tab:cl2std}.
    Irak oraz Iran otrzymały własny klaster~\ref{tab:cl8std}.
    Wyróżnia się mniejsza grupa~\ref{tab:cl7std} w której znalazły się między innymi: Chiny, Francja, Niemcy, Rosja, Wielka Brytania.


    \chapter{Podsumowanie}
    W niniejszej części zostaną opisane wnioski z pracy według kolejności wcześniej przedstawionych rozdziałów.


    \section{Dalsze kierunki rozwoju}
    W tej części zostaną opisane możliwe kierunki rozwoju pracy.

    \paragraph{Analiza pod kątem COVID-19}

    \paragraph{Analiza pod kątem grup etnicznych}

    \paragraph{Analiza pod kątem grup religijnych}


    \inputencoding{utf8}

    \newpage
    \addcontentsline{toc}{chapter}{Bibliografia}
    \printbibliography[title={Bibliografia}]

\end{document}
